\chapter{Náš algoritmus}

V tejto kapitole popíšeme náš prístup k problému opravovania dlhých čítaní. Algoritmus nepotrebuje doplňujúcu informáciu v podobne krátkych čítaní s nízkou chybovosťou. Ďalšou základnou vlastnosťou algoritmu je, že sa úplne vyhne zarovnávaniu čítaní, ktoré je jeden z časovo najnáročnejších podproblémov štandardných metód korekcie. 

Celá korekcia v našom algoritme prebieha v týchto krokoch:
\begin{enumerate}
\item Získame informáciu o vzájomnom prekrytí čítaní v genóme
\item Na čítaniach označíme úseky, ktoré sú s vysokou pravdepodobnosťou správne
\item Čítanie s čo najväčšou zhodou vyskladáme zo správnych úsekov -- svojich a tých z prekrývajúcich sa čítaní
\end{enumerate}

\input jadra_korekcie.tex

\input graf_jadier.tex

\input rekonstrukcia.tex